%%This is a very basic article template.
%%There is just one section and two subsections.
\documentclass{article}
\usepackage[latin1]{inputenc} %coding of writteninput %latin1 allows for Umlaute
\usepackage[T1]{fontenc}%vectorized fonts (cm-super package)
\usepackage[german]{babel}%some specifics of the german language
\usepackage{amsfonts, amsmath, amsthm, amssymb, paralist}

  \usepackage{listings}
\usepackage{geometry}
  \geometry{a4paper, top=25mm, left=20mm, right=15mm, bottom=20mm, headsep=10mm, footskip=12mm}
 \usepackage{rotating} 
 %Decisiontree
 \usepackage{tikz}
\usetikzlibrary{arrows.meta}
  
\usepackage{graphicx} 

\usepackage{verbatim}%f�r txt datei
\usepackage{pdfpages}
\usepackage[section]{placeins}
\title{Assignment 3}
\author{Marcel Fassbender, Miriam Wagner}

%Algorithm
%https://tex.stackexchange.com/questions/163768/write-pseudo-code-in-latex
\usepackage{algorithm}
\usepackage{algorithmic}

\begin{document}
 \setlength{\parindent}{0em} 
\maketitle
\section*{Exercise 1}
% In the previous tasks, you created RESTful APIs for various use cases. Now, let us turn to client-side
% usage of RESTful APIs. Specifically, create a social Instagram bot that does the following things:
% � It follows every account that posts something with the hashtag #SocialComputing in it.
% � It saves all posts that contain the above hashtag and were posted at the RWTH Aachen University
% (location ID 234451301) in a database.
% To simplify things, poll the REST API every 10 minutes. The API documentation for tags is
% available at https://www.instagram.com/developer/endpoints/tags and for locations at https:
% //www.instagram.com/developer/endpoints/locations/. Write the pseudo-code/algorithm for the
% follow and save actions, but list the real Instagram API URLs and JSON object paths. You can use
% a login() function that returns the access token. We are aware, that the Instagram REST API is
% deprecated; use it nevertheless. We are talking about standard Instagram posts, not stories as in the
% last exercise.
\begin{algorithm}
\caption{InstaBot}
\begin{algorithmic}[1]
\STATE{InstaBot(login,password)}
\STATE{follow(user)}
\STATE{save(post)}
\end{algorithmic}
\end{algorithm}

\begin{algorithm}
\caption{InstaBot(login=``username",password="password")}
\begin{algorithmic}[1]
\WHILE{true}
\STATE{time.sleep(10 * 60)}%10minuten warten
\STATE{all
=
Get(https://api.instagram.com/v1/tags/\{SocialComputing\}/media/recent?access\_token=ACCESS-TOKEN
)}
%https://api.instagram.com/v1/tags/search?q=snowy&access_token=ACCESS-TOKEN
\FOR{a in all}
\STATE{follow(a.user)}
\IF{234451301 == a.location}
\STATE{save(a)}
\ENDIF
\ENDFOR
\ENDWHILE
\end{algorithmic}
\end{algorithm}

\begin{algorithm}
\caption{follow(user)}
\begin{algorithmic}[1]
\STATE{self.follow(user)}
\end{algorithmic}
\end{algorithm}

\begin{algorithm}
\caption{save(post)}
\begin{algorithmic}[1]
\STATE{store in database}
\end{algorithmic}
\end{algorithm}

The Bot has to check all 10Minutes, if there happens something new. Than it has
to decide, if it is from the RWTH or not. If it is, it also has to been saved,
otherwise just followed.

\end{document}
